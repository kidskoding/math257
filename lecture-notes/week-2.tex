\section{Gaussian Elimination}
The idea behind \textbf{Gaussian Elimination} is to solve linear systems
for the pivot variables in terms of free variables (if any) in the equation
\\
Specifically, Gaussian Elimination is an \textbf{algorithm} or process used
to solve linear systems backed behind matrices
\begin{enumerate}
  \item Write down the augmented matrix
  \item Find the RREF (reduced row echelon form) of the matrix
  \item Write down the equations corresponding to the RREF
  \item Express pivot variables in terms of free variables
\end{enumerate}
\subsection{Example}
Find the general solution of
\[
\begin{aligned}
  &3x_1 - 7x_2 + 8x_3 - 5x_4 + 8x_5 = 9 \\
  &3x_1 - 9x_2 + 12x_3 - 9x_4 + 6x_5 = 15
\end{aligned}
\]
The solution simply involves following the Gaussian Elimination algorithm
\begin{enumerate}
  \item Write down the augmented matrix
    \[
    \left[
      \begin{array}{ccccc|c}
        3 & -7 & 8 & -5 & 8 & 9 \\
        3 & -9 & 12 & -9 & 6 & 15
      \end{array}
    \right]
    \]
  \item Find the RREF of the matrix
    \[
    \left[
      \begin{array}{ccccc|c}
        1 & 0 & -2 & 3 & 5 & -4 \\
        0 & 1 & -2 & 2 & 1 & -3
      \end{array}
    \right]
    \]
  \item Write down equations corresponding to the RREF
    \[
      \begin{aligned}
        &x_1 = -2x_3 + 3x_4 + 5x_5 = -4 \\
        &x_2 = -2x_3 + 2x_4 + x_5 = -3
      \end{aligned}
    \] 
  \item Express pivot variables in terms of free variables
    \[
      \begin{aligned}
        &x_1 = 2x_3 - 3x_4 - 5x_5 - 4 \\
        &x_2 = 2x_3 - 2x_4 - x_5 - 3 \\
        &x_3, x_4, x_5 = \text{free}
      \end{aligned}
    \]
\end{enumerate}
\subsection{Consistent Linear Systems}
A linear system is \textbf{consistent} if and only if an echelon form of
the augmented matrix has no row of the form $\left[\begin{array}{ccc|c} 0 
& \cdots & 0 & b \end{array}\right]$, where $b$ is nonzero. \\\\
Linear systems are consistent when
\begin{enumerate}
  \item There is a \textbf{unique} solution (no free variables)
  \item Infinitely \textbf{many} solutions (at least one free variable)
\end{enumerate}
\subsubsection{Example}
If a linear system has an augmented matrix of
$\left[\begin{array}{cc|c}
  3 & 4 & -3 \\
  3 & 4 & -3 \\
  6 & 8 & -5
\end{array}\right]$, what can be inferred about the number of solutions in the
system? \\\\\\
\textbf{Solution} \\\\
Convert matrix to echelon form
$\left[\begin{array}{cc|c}
  3 & 4 & -3 \\
  3 & 4 & -3 \\
  6 & 8 & -5
\end{array}\right] \rightarrow
\left[\begin{array}{cc|c}
  3 & 4 & -3 \\
  0 & 0 & 0 \\
  0 & 0 & 1
\end{array}\right]$ \\\\\\
There is no solution because of the $\left[\begin{array}{cc|c}
  0 & 0 & 1
\end{array}\right]$ row (the bottom row in the echelon form of
the original matrix) \\\\
As a linear equation, this row is equivalent to $0x_1 + 0x_2 = 1$, 
which is an equation that has no solution!

\section{Linear Combinations}
Consider the $m \times n$ matrices $A = \begin{bmatrix}
    a_{11} & a_{12} & \dots & a_{1n} \\
    a_{21} & a_{22} & \dots & a_{2n} \\
    \dots & \dots & \ddots & \dots \\
    a_{m1} & a_{m2} & \dots & a_{mn}
  \end{bmatrix}$, and $B = \begin{bmatrix}
    b_{11} & b_{12} & \dots & b_{1n} \\
    b_{21} & b_{22} & \dots & b_{2n} \\
    \dots & \dots & \ddots & \dots \\
    b_{m1} & b_{m2} & \dots & b_{mn}
  \end{bmatrix}$
\subsection{Sum and Scalar Product}
The \textbf{sum} of $A + B$ would be 
\[
  \begin{bmatrix}
    a_{11} + b_{11} & a_{12} + b_{12} & \dots & a_{1n} + b_{1n} \\
    a_{21} + b_{21} & a_{22} + b_{22} & \dots & a_{2n} + b_{2n} \\
    \dots & \dots & \ddots & \dots \\
    a_{m1} + b_{m1} & a_{m2} + b_{m2} & \dots & a_{mn} + b_{mn}
  \end{bmatrix}
\] 
Similarly, the \textbf{product} $cA$ for a scalar $c$ is
\[
  \begin{bmatrix}
    ca_{11} & ca_{12} & \dots & ca_{1n} \\
    ca_{21} & ca_{22} & \dots & ca_{2n} \\
    \dots & \dots & \ddots & \dots \\
    ca_{m1} & ca_{m2} & \dots & ca_{mn}
  \end{bmatrix}
\]
\subsubsection{Adding Matrices Example}
\[
  \begin{bmatrix}
    1 & 0 \\
    5 & 2
  \end{bmatrix} + 
  \begin{bmatrix}
    2 & 3 \\
    3 & 1
  \end{bmatrix} =
  \begin{bmatrix}
    1 + 2 & 0 + 3 \\
    5 + 3 & 2 + 1
  \end{bmatrix} = 
  \begin{bmatrix}
    3 & 3 \\
    8 & 3
  \end{bmatrix}
\]
\subsubsection{Scalar Multiplying Matrices Example}
\[
  5 \begin{bmatrix}
    2 & 1 & 0 \\
    3 & 1 & -1
  \end{bmatrix} = \begin{bmatrix}
    5 \cdot 2 & 5 \cdot 1 & 5 \cdot 0 \\
    5 \cdot 3 & 5 \cdot 1 & 5 \cdot -1
  \end{bmatrix} = \begin{bmatrix}
    10 & 5 & 0 \\
    15 & 5 & -5
  \end{bmatrix}
\]
\subsection{Row and Column Vectors}
\textbf{Column Vectors} are $m \times 1$-matrices, while \textbf{Row Vectors}
are $1 \times n$-matrices
\subsubsection{Examples of Row and Column Vectors}
\[
  \begin{bmatrix}
    1 \\
    2
  \end{bmatrix} \quad
  \begin{bmatrix}
    1 & 2
  \end{bmatrix} \quad
  \begin{bmatrix}
    0 \\
    -2 \\
    3
  \end{bmatrix} \quad
  \begin{bmatrix}
    1 & 2 & 3 & 4 & 5 & 6 & 7 & 8 & 9
  \end{bmatrix}
\]
\subsection{Transpose}
If $A$ is $m \times n$, the \textbf{transpose} of $A$ is the $n \times m$ matrix,
denoted by $A^T$, whose columns are formed from the corresponding rows of $A$. In 
terms of matrix elements: $(A^T)_{ij} = A_{ji}$ \\
\subsubsection{Figuring out the Transpose of a Matrix}
What is the transpose of $A = \begin{bmatrix}
  1 & 2 \\
  3 & 4 \\
  5 & 6
\end{bmatrix}$? \\\\
In a transpose, the rows of the original matrix simply become the columns of the 
transposed matrix \\\\
\[
  A^T = \begin{bmatrix}
    1 & 3 & 5 \\
    2 & 4 & 6
  \end{bmatrix}
\]
\subsection{Span}
The \textbf{linear combination} of $m \times n$-matrices $A_1, A_2, \dots, A_p$
with \textbf{coefficients} $c_1, c_2, \dots, c_p$ is defined as
\[
  c_1A_1 + c_2A_2 + \dots + c_pA_p
\] 
For example, for $m \times n$-matrices $A_1$ and $A_2$, some examples of linear combinations of these two matrices are
\[
  3A_1 + 2A_2 \quad A_1 - 2A_2 \quad \frac{1}{3}A_1 = \frac{1}{3}A_1 + 0A_2
\]
The \textbf{span} $(A_1, \dots, A_p)$ is defined as the set of all 
linear combinations of $A_1, \dots, A_p$, or simply
\[
  \text{span}(A_1,\dots,A_p) := \{c_1A_1 + c_2A_2 + \dots + c_pA_p : 
  c_1, \dots, c_p \text{ scalars} \}
\] \\\\ 
The \textbf{set} of all column vectors of length m is represented
as $\mathbb{R}^m$ \\
For example, let $a_1 = \begin{bmatrix}
  1 \\ 0 \\ 3
\end{bmatrix}, a_2 = \begin{bmatrix}
  4 \\ 2 \\ 14
\end{bmatrix}$, and $b = \begin{bmatrix}
  -1 \\ 8 \\ -5
\end{bmatrix}$. Is $b$ a linear combination of $a_1, a_2$? \\\\\\
Find $x_1, x_2$ such that
\[
  x_1\begin{bmatrix}
    1 \\ 0 \\ 3
  \end{bmatrix} + x_2\begin{bmatrix}
    4 \\ 2 \\ 14
  \end{bmatrix} = \begin{bmatrix}
    -1 \\ 8 \\ -5
  \end{bmatrix}
\] \\
\begin{enumerate}
  \item Form the system of equations
    \[
      \begin{aligned}
        &x_1 + 4x_2 = -1 \\
        &0x_1 + 2x_2 = 8 \\
        &3x_1 + 14x_2 = -5
      \end{aligned}
    \]
  \item Form the augmented matrix
    \[ \left[
      \begin{array}{cc|c}
        1 & 4 & -1 \\
        0 & 2 & 8 \\
        3 & 14 & -5
      \end{array}
    \right] \]
  \item Compute the echelon form
    \[ \left[
      \begin{array}{cc|c}
        1 & 4 & -1 \\
        0 & 2 & 8 \\
        0 & 2 & -2
      \end{array}
    \right] \rightarrow \left[ 
      \begin{array}{cc|c}
        1 & 4 & -1 \\
        0 & 2 & 8 \\
        0 & 0 & -10
      \end{array} \right]
    \]
\end{enumerate}
With this echelon form matrix, it can be concluded that $b$ is not a linear
combination of $a_1$ and $a_2$ because the system is inconsistent \\\\
This means that there is no solution that exists for the bottom row linear equation
\[
  0x_1 + 0x_2 = -10
\]
Therefore, this means that there is no solution that exists for the 
derived linear system
\textbf{and} the original vector equation,  
  $x_1\begin{bmatrix}
    1 \\ 0 \\ 3
  \end{bmatrix} + x_2\begin{bmatrix}
    4 \\ 2 \\ 14
  \end{bmatrix} = \begin{bmatrix}
    -1 \\ 8 \\ -5
  \end{bmatrix}$ \\\\
Geometrically speaking, this means that $b$ is not in the span of $a_1$ and $a_2$
\subsection{TTK - Things to Know}
Solving linear systems is the same as finding linear combinations! \\\\
A vector equation
\[
  x_1a_1 + x_2a_2 + \cdots + x_na_n = b
\]
has the same solution set as the linear system whose augmented matrix is
\[ \left[
  \begin{array}{cccc|c}
    a_1 & a_2 & \cdots & a_n & b
  \end{array}
\right] \]
In particular, $b$ can be generated by a linear combination of $a_1, a_2, \cdots,
a_n$ if and only if there is a solution to the linear system corresponding to
the augmented matrix \\\\
A matrix is defined in terms of its \textbf{colums} or \textbf{rows}
\[
  A := \begin{bmatrix}
      a_1 & a_2 & \cdots & a_n
  \end{bmatrix} \quad \text{or}
  A := \begin{bmatrix}
      R_1 \\ R_2 \\ \vdots \\ R_m
  \end{bmatrix}
\]

\section{Matrix Vector Multiplication}
Suppose $x$ is a vector in $\mathbb{R}^m$ and $A = \begin{bmatrix}
  a_1 & \dots & a_n
\end{bmatrix}$ an $m \times n$-matrix. The product $Ax$ is defined by
\[
  Ax = x_1a_1 + x_2a_2 + \dots + x_na_n
\]
\begin{itemize}
  \item $Ax$ is a linear combination of the columns of $A$ using the entries 
    in $x$ as coefficients.
  \item $Ax$ is only defined if the number of entries of $x$ is equal to
    the number of columns of $A$
\end{itemize}
\subsection{Example Problem}
If $A = \begin{bmatrix}
  2 & 0 \\
  1 & 1
\end{bmatrix}$, $B = \begin{bmatrix}
  1 & 2 \\
  0 & 1 \\
  3 & 5
\end{bmatrix}$, and $x = \begin{bmatrix}
  2 \\ 3
\end{bmatrix}$, determine $Ax$ and $Bx$
\\\\[8pt]
\[
  Ax = \begin{bmatrix}
    2 & 0 \\
    1 & 1
  \end{bmatrix} \begin{bmatrix}
    2 \\ 3
  \end{bmatrix} = 2 \begin{bmatrix}
    2 \\ 1
  \end{bmatrix} + 3 \begin{bmatrix}
    0 \\ 1
  \end{bmatrix} = \begin{bmatrix}
    4 \\ 2
  \end{bmatrix} + \begin{bmatrix}
    0 \\ 3
  \end{bmatrix} = \begin{bmatrix}
    4 \\ 5
  \end{bmatrix}
\]
\[
  Bx = \begin{bmatrix}
    1 & 2 \\
    0 & 1 \\
    3 & 5
  \end{bmatrix} \begin{bmatrix}
    2 \\ 3
  \end{bmatrix} = 2 \begin{bmatrix}
    1 \\ 0 \\ 3
  \end{bmatrix} + 3 \begin{bmatrix}
    2 \\ 1 \\ 5
  \end{bmatrix} = \begin{bmatrix}
    8 \\ 3 \\ 21
  \end{bmatrix}
\]
\subsection{Example Problem 2}
Consider the vector equation 
\[
  x_1 \begin{bmatrix}
    1 \\ 2
  \end{bmatrix} + x_2 \begin{bmatrix}
    3 \\ 4
  \end{bmatrix} = \begin{bmatrix}
    0 \\ 2
  \end{bmatrix}
\]
\\[8pt]
Find a $2 \times 2$ matrix $A$ such that $(x_1, x_2)$ is a solution to the above
equation if and only if
\[
  A \begin{bmatrix}
    x_1 \\ x_2
  \end{bmatrix} = \begin{bmatrix}
    0 \\ 2
  \end{bmatrix}?
\]
\\[8pt]
Take $A = \begin{bmatrix}
  1 & 3 \\
  2 & 4
\end{bmatrix}$. Form a linear equation with $A$
using a linear combination $Ax$, where $x$ represents the column vector
of unknown variables
\[
  Ax = \begin{bmatrix}
    1 & 3 \\
    2 & 4
  \end{bmatrix} \begin{bmatrix}
    x_1 \\ x_2
  \end{bmatrix} = x_1 \begin{bmatrix}
    1 \\ 2
  \end{bmatrix} + x_2 \begin{bmatrix}
    3 \\ 4
  \end{bmatrix}
\]
$x_1 \begin{bmatrix}
  1 \\ 2
\end{bmatrix} + x_2 \begin{bmatrix}
  3 \\ 4
\end{bmatrix}$ is the vector expression that matches the problem statement and
we know that it is equivalent to the column vector $\begin{bmatrix}
  0 \\ 2
\end{bmatrix}$. $A \begin{bmatrix}
  x_1 \\ x_2
\end{bmatrix}$ is also equivalent to to the column vector $\begin{bmatrix}
  0 \\ 2
\end{bmatrix}$. 
\\[8pt]
Therefore one possible $2 \times 2$ matrix is $\begin{bmatrix}
  1 & 2 \\
  3 & 4
\end{bmatrix}$
\subsection{Equivalent Formations of a Linear System}
Let $A = \begin{bmatrix}
  a_1, \cdots, a_n
\end{bmatrix}$ be an $m \times n$-matrix and b in $\mathbb{R}^m$. The the following
are equivalent
\begin{itemize}
  \item $(x_1, x_2, \dots, x_n)$ is a solution of the vector equation, $x_1a_1 + x_2a_2 + \cdots x_na_n = b$
  \item $\begin{bmatrix}
    x_1 \\
    \vdots \\
    x_n
  \end{bmatrix}$ is a solution to the matrix equation, $Ax = b$
  \item $(x_1, x_2, \dots, x_n)$ is a solution of the system with augmented
    matrix, $\left[ \begin{array}{c|c}
      A & b
  \end{array} \right]$
\end{itemize}
\\[8pt]
The notation for the system of equations with augmented matrix $\begin{bmatrix}
  A & b
\end{bmatrix}$ will be written as $Ax = b$
\subsection{Matrices as Machines}
Let $A$ be a $m \times n$ matrix.
\begin{enumerate}
  \item Input: $n$-component vector $x \in \mathbb{R}^m$
  \item Output: $m$-component vector $b = Ax \in \mathbb{R}^m$
\end{enumerate}
For example, consider the matrix $A = \begin{bmatrix}
  0 & 1 \\
  1 & 0
\end{bmatrix}$. What does this machine do?
\\[8pt]
\textbf{Solution}
\begin{enumerate}
  \item Let $x = \begin{bmatrix}
  x_1 \\ x_2
\end{bmatrix}$ be our input
\[
  Ax = \begin{bmatrix}
    0 & 1 \\ 1 & 0
  \end{bmatrix} = x_1 \begin{bmatrix}
    0 \\ 1
  \end{bmatrix} + x_2 \begin{bmatrix}
    1 \\ 0
  \end{bmatrix} = \begin{bmatrix}
    x_2 \\ x_1
  \end{bmatrix}
\]
  \item Therefore, the machine $A$ switches the entries of the vector $x$
\end{enumerate}
\\[8pt]
Geometrically speaking, this machine reflects across the $x_1 = x_2$-line
\subsection{Composition of Machines}
Let $A$ be an $m \times n$ matrix and $B$ be an $k \times l$ matrix. Now we 
can compose the two machines 
\\[8pt]
However, this composition only works for some $k, l, m, n$. For which?
\\[8pt]
\textbf{Solution}
\begin{itemize}
  \item If $A$ is an $m \times n$-matrix and $x$ in $\mathbb{R}^n$, 
  then $Ax$ is in $\mathbb{R}^m$
  \item In order to calculate $B(Ax)$ when then need $B$ to have $m$ columns.
  \item So we need $l = m$. Both $n$ and $k$ can be arbitrary.
\end{itemize}
\subsubsection{Worked Example}
Let $A = \begin{bmatrix}
  0 & 1 \\
  1 & 0
\end{bmatrix}$ and $B = \begin{bmatrix}
  1 & 0 \\
  0 & 0
\end{bmatrix}$ be as before. Is $A(Bx) = B(Ax)$?
\\[8pt]
\textbf{Solution}
No, projection and reflection do not commute!
\[
  A(B\begin{bmatrix}
    1 \\ 2
  \end{bmatrix}) = A \begin{bmatrix}
    1 \\ 0
  \end{bmatrix} = \begin{bmatrix}
    0 \\ 1
  \end{bmatrix} \quad
  B(A\begin{bmatrix}
    1 \\ 2
  \end{bmatrix}) = B \begin{bmatrix}
    2 \\ 1
  \end{bmatrix} = \begin{bmatrix}
    2 \\ 0
  \end{bmatrix}
\]

\section{Matrix Multiplication}
Let $A$ be an $m \times n$ matrix and let $B = \begin{bmatrix}
  b_1 \dots b_p
\end{bmatrix}$ be an $n \times p$-matrix. We define
\[
  AB := \begin{bmatrix}
    Ab_1 & Ab_2 & \dots & Ab_p
  \end{bmatrix}
\]
\\[8pt]
Compute AB where $A = \begin{bmatrix}
  4 & -2 \\
  3 & -5 \\
  0 & 1
\end{bmatrix}$ and $B = \begin{bmatrix}
  2 & -3 \\
  6 & -7
\end{bmatrix}$
\\[8pt]
\textbf{Solution}
\[
  &Ab_1 = \begin{bmatrix}
      4 & -2 \\
      3 & -5 \\
      0 & 1 
    \end{bmatrix} \begin{bmatrix}
      2 \\ 6
    \end{bmatrix} = 2 \begin{bmatrix}
      4 \\ 3 \\ 0 
    \end{bmatrix} + 6 \begin{bmatrix}
      -2 \\ -5 \\ 1
    \end{bmatrix} = \begin{bmatrix}
      -4 \\ -24 \\ 6
    \end{bmatrix}
\] 
\[
  Ab_2 = \begin{bmatrix}
      4 & -2 \\
      3 & -5 \\
      0 & 1
    \end{bmatrix} \begin{bmatrix}
      -3 \\ -7
    \end{bmatrix} = -3 \begin{bmatrix}
      4 \\ 3 \\ 0
    \end{bmatrix} - 7 \begin{bmatrix}
      -2 \\ -5 \\ 1
    \end{bmatrix} = \begin{bmatrix}
      2 \\ 26 \\ -7
    \end{bmatrix}
\]
These are the columns of the product matrix, where $AB = \begin{bmatrix}
  -4 & 2 \\
  -24 & 26 \\
  6 & -7
\end{bmatrix}$
\\[8pt]
Note that $Ab_1$ and $Ab_2$ are linear combinations of the columns of $A$. This 
means that each column of $AB$ is a linear combination of the columns of $A$ using
coefficients from the corresponding columns of $B$
\subsection{Worked Example}
Let $A$ be an $m \times n$ matrix and let $B$ be an $n \times p$-matrix. We define
\[
  AB := \begin{bmatrix}
    Ab_1 & Ab_2 & \dots & Ab_p
  \end{bmatrix}
\]
If $C$ is a $4 \times 3$ and $D$ is a $3 \times 2$, are $CD$ and $DC$ defined?
What are their sizes or dimensions?
\\[8pt]
\textbf{Solution}
\begin{enumerate}
  \item The product $AB$ can only be defined if $B$ has as many rows as $A$ has 
    columns
  \item If this is the case, then $AB$ has as many rows as $A$ and as many columns 
    as $B$
  \item Therefore, $CD$ is defined and has a $4 \times 2$ dimension, while 
    $DC$ is not defined
\end{enumerate}
\\\\
Recall that matrices can be thought of as machines
\begin{itemize}
  \item Let $B$ be $n \times p$: input $x \in \mathbb{R}^p$, output $c = Bx \in 
  \mathbb{R}^n$
\item Let $A$ be $m \times n$: input $y \in \mathbb{R}^n$, output $b = Ay \in 
  \mathbb{R}^m$
\end{itemize}
\\[8pt]
Compute $(AB)x$ and $A(B(x))$. Are these the same?
\\[8pt]
\textbf{Solution}
\[
  Bx = \begin{bmatrix}
    1 & 2 \\
    0 & 1
  \end{bmatrix} \begin{bmatrix}
    x_1 \\
    x_2
  \end{bmatrix} = x_1 \begin{bmatrix}
    1 \\ 0
  \end{bmatrix} + x_2 \begin{bmatrix}
    2 \\ 1
  \end{bmatrix} = \begin{bmatrix}
    x_1 + 2x_2 \\
    x_2
  \end{bmatrix}
\]
\[
  A(Bx) = \begin{bmatrix}
    2 & 0 \\
    1 & 1
  \end{bmatrix} \begin{bmatrix}
    x_1 + 2x_2 \\
    x_2
  \end{bmatrix} = (x_1 + 2x_2) \begin{bmatrix}
    2 \\ 1
  \end{bmatrix} + (x_2) \begin{bmatrix}
    0 \\ 1
  \end{bmatrix} = \begin{bmatrix}
    2x_1 + 4x_2 \\
    x_1 + 3x_2
  \end{bmatrix}
\]
\[
  (AB)x = \begin{bmatrix}
    Ab_1 & Ab_2
  \end{bmatrix} \begin{bmatrix}
    x_1 \\ x_2
  \end{bmatrix} = x_1Ab_1 + x_2Ab_2 = x_1(1\begin{bmatrix}
    2 \\ 1
  \end{bmatrix} + 0\begin{bmatrix}
    0 \\ 1
  \end{bmatrix}) + x_2(2 \begin{bmatrix}
    2 \\ 1
  \end{bmatrix} + 1 \begin{bmatrix}
    0 \\ 1
  \end{bmatrix}) 
\]
\[
= x_1 \begin{bmatrix}
    2 \\ 1
  \end{bmatrix} + x_2 \begin{bmatrix}
    4 \\ 3
  \end{bmatrix} = \begin{bmatrix}
    2x_1 + 4x_2 \\
    x_1 + 3x_2
  \end{bmatrix}
\]
\\[8pt]
Let $A$ be an $m \times n$ matrix and $B$ be an $n \times p$ matrix. Then for every $x \in \mathbb{R}^p$ 
\[
  A(Bx) = (AB)x
\]
\subsection{Row Column Rule}
Let $A$ be $m \times n$ and $B$ be $n \times p$ such that 
\[
  A = \begin{bmatrix}
    R_1 \\ \vdots \\ R_m
  \end{bmatrix}, \text{ and }
  B = \begin{bmatrix}
    C_1 & \cdots & C_p
  \end{bmatrix}
\]
Then 
\[
  AB = \begin{bmatrix}
    R_1C_1 & \cdots & R_1C_p \\
    R_2C_1 & \cdots & R_2C_p \\
    R_mC_1 & \cdots & R_mC_p
  \end{bmatrix} \text{ and }
  (AB)_{ij} = R_iC_j = a_{i1}b_{1j} + a_{i2}b_{2j} + \cdots 
  + a_{in}b_{nj}
\]
\subsubsection{Example Problem}
Let $A = \begin{bmatrix}
  2 & 3 & 6 \\
  -1 & 0 & 1
\end{bmatrix}$ and $B = \begin{bmatrix}
  2 & -3 \\
  0 & 1 \\
  4 & -7
\end{bmatrix}.$ Compute $AB$, if it is defined
\[
  AB = \begin{bmatrix}
    2 \cdot 2 + 3 \cdot 0 + 6 \cdot 4 & 2 \cdot (-3) + 3 \cdot 1 + 6 \cdot (-7) \\
    -1 \cdot 2 + 0 \cdot 0 + 1 \cdot 4 & -1 \cdot (-3) + 0 \cdot 1 + 1 \cdot (-7)
  \end{bmatrix} = \begin{bmatrix}
    28 & -45 \\
    2 & -4
  \end{bmatrix}
\]
\subsection{Outer Product Rule}
Let $A$ be $m \times n$ and $B$ be $n \times p$ such that
\[
  A = \begin{bmatrix}
    C_1 \cdots C_n
  \end{bmatrix}, \text{ and } B = \begin{bmatrix}
    R_1 \\
    \vdots \\
    R_n
  \end{bmatrix}
\]
Then 
\[
  AB = C_1R_1 + \cdots + \cdots + C_nR_n
\]
\subsubsection{Example Problem}
Let $A = \begin{bmatrix}
  2 & 3 & 6 \\
  -1 & 0 & 1
\end{bmatrix}$ and $B = \begin{bmatrix}
  2 & -3 \\
  0 & 1 \\
  4 & -7
\end{bmatrix}.$ Compute $AB$, if it is defined
\\[8pt]
\textbf{Solution}
\[
  \begin{bmatrix}
    2 & 3 & 6 \\
    -1 & 0 & 1
  \end{bmatrix} \begin{bmatrix}
    2 & -3 \\
    0 & 1 \\
    4 & -7
  \end{bmatrix} = \begin{bmatrix}
    2 \\ -1
  \end{bmatrix} \begin{bmatrix}
    2 & -3
  \end{bmatrix} + \begin{bmatrix}
    3 \\ 0
  \end{bmatrix} \begin{bmatrix}
    0 & 1
  \end{bmatrix} + \begin{bmatrix}
    6 \\ 1
  \end{bmatrix} \begin{bmatrix}
    4 & -7
  \end{bmatrix}
\]
\[
  = \begin{bmatrix}
    4 & -6 \\
    -2 & 3
  \end{bmatrix} + \begin{bmatrix}
    0 & 3 \\
    0 & 0
  \end{bmatrix} + \begin{bmatrix}
    24 & -42 \\
    4 & -7
  \end{bmatrix}
\]
\[
  = \begin{bmatrix}
    28 & -45 \\
    2 & -4
  \end{bmatrix}
\]
