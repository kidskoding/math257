\newpage
\appendix

\section{Graphs}
\textbf{Terminology}
\begin{itemize}
  \item \textbf{Node} - an individual point/entity on a graph (also known as a vertex) 
  \item \textbf{Edge} - a connection between two nodes 
  \item \textbf{Graph} - a collection of nodes and edges
\end{itemize}
A proper graph consists of a set of nodes and a set of edges, formally notated as $(G = VE)$. This means that any collection of nodes and edges is formally conisdered as a graph, where
\[
  G = VE, \text{ where V is the set of vertices, and E is the set of edges}
\]

\begin{figure}[htbp]
  \centering
  \begin{tikzpicture}[
      every node/.style={circle, draw, minimum size=6mm, inner sep=0pt},
      node distance=8mm and 12mm
    ]
    % Nodes
    \node (1) {1};
    \node (2) [right=of 3] {2};
    \node (3) [below=of 4] {3};
    \node (4) [left=of 5]  {4};
    \node (5) [above right=of 2] {5};
    \node (6) [above left=of 4] {6};

    % Edges
    \draw (4) -- (6);
    \draw (4) -- (1);
    \draw (4) -- (3);
    \draw (3) -- (2);
    \draw (2) -- (5);
  \end{tikzpicture}
  \caption{A small undirected graph on six nodes.}
  \label{fig:tikz-graph}
\end{figure}
\subsection{Adjacency Matrices}
Consider Figure 1, which shows a graph of 6 nodes and 5 edges. Like any other graph, it can be represented as an adjacency matrix, $A = (a_{ij})$ 
\[
  a_{ij} = \begin{cases} 
    1 & i, j \text{ share common edge} \\
    0 & \text{else}
  \end{cases}
\]
\textbf{Adjacency Matrix} - a square matrix used to represent the nodes 
and their connections on any given graph\\\\
$a_{ij}$ tells us that there exists an edge between two nodes when nodes $i$ and $j$ 
share a common edge. Or mathematically speaking, 
\[
  a_{ij} = 1 \leftrightarrow \exists e \in E : e = \left\{i, j\right\}, 
  \text{ where $E$ denotes the set of edges}
\]
\newpage 
In the context of Figure 1, we can form an adjancency matrix based on the edges that 
connect our nodes, by following a series of steps. 
\begin{enumerate}
  \item Identify the edges 
  \item Create a square matrix that orders the rows and columns based on the number of 
    nodes that exist in the graph. 
  \begin{itemize}
    \item $i$ represents the row node and $j$ represents the column node
    \item Write $0$ if there is no edge that connects between $i$ and $j$, or $1$ 
      if there is an edge that connects between $i$ and $j$ 
  \end{itemize}
\end{enumerate}
By following these steps, the graph from figure 1 results in an Adjacency Matrix, $A$
where 
\[
  A = \begin{bmatrix}
    0 & 0 & 0 & 1 & 0 & 0 \\
    0 & 0 & 1 & 0 & 1 & 0 \\
    0 & 1 & 0 & 1 & 0 & 0 \\
    1 & 0 & 1 & 0 & 0 & 1 \\
    0 & 1 & 0 & 0 & 0 & 0 \\
    0 & 0 & 0 & 1 & 0 & 0 
  \end{bmatrix}
\]
\subsection{Walks and Paths}
A \textbf{walk} consists of a sequence of nodes, $v_0, v_1, v_2, \dots$, where $v_i, v_{i + 1}, \dots$ share an edge and a \textbf{path} consists of a walk with distinct nodes 
in $G = (V, E)$ \\\\
In a sequence of vertices, $W = (v_0, v_1, \dots, v_k)$, a \textbf{walk} exists  
when there exists an edge connecting two nodes in every step along $W$
\[
  W \text{ is a walk } \leftrightarrow \forall i \in \{1, \dots, k\}, \exists e \in 
  E : e = \{v_{i - 1}, v_i\}
\] 
Similarly speaking, let $P = (v_0, v_1, \dots, v_k)$ represent a sequence of vertices 
in $G = (V, E)$. Then $P$ is a path of length $k$ if and only if $v_i \neq v_j$
\[
  P \text{ is a path } \leftrightarrow (\forall 1 \leq i \leq k, \{v_{i - 1}, 
  v_i\} \in E) \text{ and } (v_i \neq v_j) \text{ for all } 0 \leq i < j \leq k
\]
\subsection{Directed Graph}
\textbf{Directed Graph} - a graph where each edge has its own orientation/direction. 
\begin{center}
\begin{tikzpicture}[->, >=Stealth, node distance=2cm, every node/.style={circle, draw, minimum size=1cm}]
  \node (A) {A};
  \node (B) [right of=A] {B};
  \node (C) [below of=A] {C};
  \node (D) [below of=B] {D};

  \draw (A) -- (B);
  \draw (A) -- (C);
  \draw (B) -- (D);
  \draw (C) -- (D);
  \draw (D) to[bend left] (A);
\end{tikzpicture}
\end{center}
The graph above represents a directed graph, with nodes $A$, $B$, $C$, and $D$ with 
directed edges that point to other nodes. 
\subsubsection{Edge-Node Incidence Matrices}
Since the edges are directed, we can describe how the edges connected 
to the nodes through an edge-node incidence matrix (or simply an incidence matrix). \\\\
\textbf{Incidence Matrix} - A matrix that describes how edges are connected to nodes on a graph, specifically speaking a directed graph \\\\ 
In an incidence matrix, there exists an adjacency matrix, $a_{ij}$, 
with the values $-1$, $0$, and $1$, where
\[
  a_{ij} = \begin{cases}
    -1 & \text{ edge $i$ leaves node $j$} \\
    +1 & \text{ edge $i$ enters node $j$} \\
    0 & \text{ otherwise}
  \end{cases}
\]
This forms the incidence matrix, $A$, where 
\[
  A = \begin{bmatrix}
    -1 & 1 & 0 & 0 \\
    -1 & 0 & 1 & 0 \\
    0 & 1 & -1 & 0 \\
    0 & -1 & 0 & 1
  \end{bmatrix}
\]
\subsection{Proving Graph Theorems}
Given a directed graph, $G$ and its edge-node incidence matrix, $A$, 
then dim(Nul($A$)) = the number of connected components. \\\\
We can prove this theorem by using the 0 vector, where
\[
  \vec{0} = \vec{A}x = A \begin{bmatrix} x_1 \\ x_2 \\ x_3 \\ x_4 \end{bmatrix} = \\
  \begin{bmatrix} -x_1 + x_2 \\ -x_1 + x_3 \\ -x_2 + x_4 \\ -x_3 + x_4 \end{bmatrix}
\]
Or simply speaking, 
\[
  \vec{0} = 
  \begin{bmatrix} -x_1 + x_2 \\ -x_1 + x_3 \\ x_2 - x_3 \\ -x_2 + x_4 \\ -x_3 + x_4 \end{bmatrix}
\]
With this, we can derive the following solutions 
\[
\begin{aligned}
  x_1 = x_2 \\
  x_1 = x_3 \\ 
  x_2 = x_3 \\
  x_2 = x_4 \\
  x_3 = x_4
\end{aligned}
\]
Where, the Nul($A$) = span($\begin{bmatrix} 1 \\ 1 \\ 1 \\ 1 \end{bmatrix}$) 
\subsubsection{Example Proof}
Given the directed graph, $G$, where $G$ represents the graph below 
\begin{center}
\begin{tikzpicture}[->, >=Stealth, node distance=2cm, every node/.style={circle, draw, minimum size=1cm}]
  \node (1) {1};
  \node (3) [below of=1] {3};
  \node (2) [right of=1] {2};
  \node (4) [below of=2] {4};

  \draw (1) -- (3);
  \draw (2) -- (4);
\end{tikzpicture}
\end{center}
There are two connected components, where $Nul(A)$ forms the basis $B$, where $B$ =  
\{$\begin{bmatrix}1 \\ 0 \\ 1 \\ 0\end{bmatrix}, \begin{bmatrix}0 \\ 1 \\ 0 \\ 1 
\end{bmatrix}$\}. \\\\
We can prove this basis by forming the resulting matrix and solving for its variables by setting it equal to the zero vector, $\vec{0}$
\[
  \begin{aligned}
    &A\vec{x} = \vec{0} \\ 
    &A = \begin{bmatrix}
        -1 & 0 & 1 & 0 \\ 
        0 & -1 & 0 & 1
    \end{bmatrix} \\
    &\vec{0} = \begin{bmatrix}
      -x_1 + x_3 \\
      -x_2 + x_4 
    \end{bmatrix} \\
    &x_1 = x_3 \\ 
    &x_2 = x_4
  \end{aligned}
\]
\subsection{Cycles, Cycle Vectors, and Cycle Spaces}
A \textbf{cycle} is a closed path in a graph, where its sequence of vertices and edges include the same vertex at the start and end of said sequence. Only the first and last vertices are equal \\\\
Consider the graph below,
\begin{center}
\begin{tikzpicture}[->, >=Stealth, node distance=2cm, every node/.style={circle, draw, minimum size=1cm}]
  \node (A) {A};
  \node (B) [right of=A] {B};
  \node (C) [below of=A] {C};
  \node (D) [below of=B] {D};

  \draw (A) -- (B);
  \draw (A) -- (C);
  \draw (B) -- (D);
  \draw (C) -- (D);
  \draw (D) to[bend left] (A);
\end{tikzpicture}
\end{center}
\textbf{Cycle Vector} - A vectorized representation of a cycle in a graph, which records which edges are part of that cycle and in which direction they are traversed \\\\
Recall $a_{ij}$, where $a_{ij}$ represents the connection between 
nodes $i$ and $j$
\[
  a_{ij} = \begin{cases}
    -1 & \text{ edge $i$ leaves node $j$} \\
    +1 & \text{ edge $i$ enters node $j$} \\
    0 & \text{ otherwise}
  \end{cases}
\]
A cycle, $c$, exists between $A \rightarrow C \rightarrow D \rightarrow A$. Its incidence matrix,  
\[
  A = \begin{bmatrix} -1 & 1 & 0 & 0 \\ -1 & 0 & 1 & 0 
\\ 0 & 1 & -1 & 0 \\ 0 & -1 & 0 & 1 \end{bmatrix} 
\]. Given the resulting cycle vector for $c$ 
\[
  c = \begin{bmatrix} 0 \\ 0 \\ 1 \\ 1 \\ 1 \end{bmatrix}
\] 
By deriving the cycle vectors for every cycle, we get the \textbf{cycle space}, which is the span of cycle vectors. \\\\
Assume a generic graph $G$ is a directed graph with an incidence matrix $A$. Prove that the cycle space of $G$ = $Nul(A^T)$ \\\\
Recall that the null space of a matrix is the set of all vectors, $\vec{x}$, such that $A\vec{x} = 0$. 
\[
  \vec{0} = A^T\vec{y} = \begin{bmatrix} -1 & -1 & 0 & 0 & 0 \\ 
    1 & 0 & 1 & -1 & 0 \\ 0 & 1 & -1 & 0 & -1 \\ 0 & 0 & 0 & 1 & 1 
  \end{bmatrix}\vec{y}
\]
And therefore, the $RREF_{A}$, reduced row echelon form of $A$, is equal to 
\[
  \begin{bmatrix}
    -1 & -1 & 0 & 0 & 0 \\
    0 & -1 & 1 & -1 & 0 \\ 
    0 & 0 & 0 & -1 & -1 \\
    0 & 0 & 0 & 0 & 0
  \end{bmatrix}
\]
Use $RREF_A$ to form the solution vector, which results in 
\[
  \begin{bmatrix}
    r \\
    s \\ 
    -s \\ 
    r + s \\
    -r - s \\
  \end{bmatrix}
\]
This forms a nullspace, where Nul($A^T$) = \{\begin{bmatrix}
  -r - s \\ 
  r + s \\ 
  r \\ 
  -s \\
  s
\end{bmatrix}, where $r, s \in \mathbb{R}$\}, which results in the 
cycle space of 
\[
  span\{\begin{bmatrix} -1 \\ 1 \\ 1 \\ 0 \\ 0 \end{bmatrix}, 
  \begin{bmatrix} -1 \\ 1 \\ 0 \\ -1 \\ 1 \end{bmatrix}\}
\]

\section{Linear Differential Equations}
