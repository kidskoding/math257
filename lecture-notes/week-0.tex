\section{MATH 257 - Course Overview}
MATH 257 is an introductory linear algebra course administered by the \textbf{University of Illinois Urbana-Champaign (UIUC)}, which covers basic definitions and algorithms of the subject needed in higher levels of engineering, science, and economics. \\\\
This course has a bulk emphasis on introducing the mathematical theory behind the field 
of Linear Algebra, along with a gentle introduction to how Linear Algebra can be applied to various computer programs and algorithms. This includes understanding 
how to implement various Linear Algebra related concepts 
and applying them to real world scenarios in the \textbf{Python programming language} \\\\
\textbf{Therefore, prior Python experience is recommended} \\\\
\textbf{This document covers MATH 257 for the Spring 2025 semester!}

\subsection{Prequisites, Future Courses, and MATH 257 Resources}
\href{https://lti.learn.illinois.edu/pluginfile.php/3027342/mod_resource/content/43/s25-math257--Syllabus.pdf}{The \textbf{syllabus} for MATH 257 Spring 2025} \\\\
\textbf{Prerequisites}
\begin{itemize}
  \item MATH 220/MATH 221 (Calculus I)
  \item CS 101 (Introduction to Computing) - \textbf{note that any other equivalent 
    computing course can also be completed},
  \begin{itemize}
    \item For CS majors, this can include CS 124 (Introduction to Computer Science I) or CS 128 (Introduction to Computer Science II)
    \item For ECE majors, this can include ECE 120 (Intro to Computing) or ECE 220 (Computer Systems and Programming)
  \end{itemize}
\end{itemize}
\textbf{Future Courses} - CS Majors may take CS 357 (Numerical Methods), which expands
upon the computational application of the concepts taught in this course. ECE majors 
can take the equivalent MATH 357.

\subsection{About the Author}
Hello, my name is Anirudh Konidala and I am a UIUC student studying Computer Science 
and Education. MATH 257 was definitely a big struggle for me and I never fully studied the material well enough to ace the midterms \\\\ 
Therefore, when it came to the last few weeks before the MATH 257 final exam, I decided to compile a big set of notes on both the \textbf{mathematical} 
and \textbf{programming} portion of this course, so that I could ace this course 
and pass the course with a decent grade. \\\\ 
I hope this content can also help others ace MATH 257, so that they don't 
have to bomb the midterms/exams like I did.

\subsection{Overview}
Linear algebra is the branch of mathematics that deals with vector spaces and linear transformations between them. It focuses on concepts like vectors, matrices, systems of linear equations, determinants, eigenvalues, and eigenvectors.

\subsection{Book Structure}
This book is structured with regards to the Lecture Videos and Modules from Canvas. 
There are 46 chapters, each representing various essential Linear Algebra concepts 
taught in MATH 257. Each section corresponds to the appropriate Module lecture video 
on Canvas. 
If there is a lack of Canvas access, you can view the module videos on \href{https://mediaspace.illinois.edu/playlist/dedicated/1_sbehz0wr/1_nkdnuuw8}{Mediaspace}. Each chapter/module is the contents/table of contents. 
\begin{itemize}
\item Midterm 1 - Chapters/Modules 1 - 11
\item Midterm 2 - Chapters/Modules 12 - 23
\item Midterm 3 - Chapters/Modules 24 - 38
\item Final Exam - Chapters/Modules 1 - 46
\end{itemize}
\newpage
\tableofcontents
\newpage
