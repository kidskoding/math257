\section{Introduction to Linear Systems}
\textbf{Linear Equations} are in the form of
\[
  a_{1}x_{1} + \dots + a_{n}x_{n} = b
\]
where $a_1,\dots,a_n,b$ are numbers and $x_1,\dots,x_n$ are variables. \\\\
For Example,
\[
  4x_1 - 5x_2 + 2 = x_1
\]
is a linear equation because it can be rearranged to form an equation that is in the form of $a_{1}x_{1} + \dots + a_{n}x_{n} = b$
\[
\begin{aligned}
  4x_1 - 5x_2 + 2 &= x_1 \\
  4x_1 - x_1 - 5x_2 &= -2 \\
  3x_1 - 5x_2 &= -2 \\
\end{aligned}
\]
However,
\[
  x_2 = 2\sqrt{x_1} - 7
\]
is \textbf{not} a linear equation because it cannot be expressed in the form of $a_{1}x_{1} + \dots + a_{n}x_{n} = b$ \\
\textbf{Linear Systems} are collections of one or more linear equations involving the same set of variables, say, $x_1, x_2, \dots, x_n$. \\\\
A \textbf{solution} of a linear system is a list $(s_1,s_2,\dots,s_n)$ of numbers that makes each equation in the system true when the values $s_1, s_2, \dots, s_n$ are substituted for $x_1, x_2, \dots, x_n$, respectively.
\subsection{Example Problem - Two equations in two variables}
\[
  \begin{aligned}
    x_1 + x_2 &= 1 \\
    -x_1 + x_2 &= 0
  \end{aligned}
\]
What is a solution for this system of linear equations? \\\\
\textbf{Solution} - Use the \textbf{elimination method}
\begin{enumerate}
  \item Add the two systems to eliminate the $x_1$ variable
  \[
    \begin{aligned}
      &2x_2 = 1 \\
      &x_2 = \frac{1}{2}
    \end{aligned}
  \]
  \item Plug into the first equation to find the $x_2$ variable
  \[
    \begin{aligned}
      &x_1 + \frac{1}{2} = 1 \\
      &x_1 = \frac{1}{2}
    \end{aligned}
  \]
  \item Thus $(x_1, x_2) = (\frac{1}{2},\frac{1}{2})$ is the only solution. 
\end{enumerate}
\subsection{Does every system have a solution?}
\textbf{No!} Observe the system
\[
\begin{aligned}
  x_1 - 2x_2 &= -3 \\
  2x_1 - 4x_2 &= 8
\end{aligned}
\] \\\\
Its process of solving is as follows
\begin{enumerate}
  \item Multiply the first equation by 2 to eliminate the $x_1$ variable
    \[
      2x_1 - 4x_2 = -6
    \]
  \item Subtract the first equation from the second to cancel $x_1$
    \[
      0 = 14
    \]
  \item The equation $0 = 14$ is always false, so \textbf{no solutions} exist.
\end{enumerate}
\subsubsection{Example Problem}
\[
  \begin{aligned}
    x_1 + x_2 &= 3 \\
    -2x_1 - 2x_2 &= -6
  \end{aligned}
\]
How many solutions are there to this system of equations? \\\\
\textbf{Solution}
\begin{enumerate}
  \item Multiply the first equation by $-2$ to eliminate $x_1$
    \[
      -2x_1 - 2x_2 = -6
    \]
  \item Both the first and second equation are the same. Subtracting the two in order to cancel out $x_1$ will result in
    \[
      0 = 0
    \]
  \item This means both equations have the same solutions. Therefore, the system is said to have \textbf{infinitely many solutions}.
\end{enumerate}
\subsection{Theorem 1}
A linear system has either \textbf{one unique solution}, \textbf{no solution}, or \textbf{infinitely many solutions}. \\\\
\textbf{Definition} - the \textbf{solution set} of a linear system is the set of all solutions of the linear system. Two linear systems are \textbf{equivalent} if they have the same solution set. \\\\
The general strategy is to replace one system with an equivalent system that is easier to solve.
\subsubsection{Example Problem}
Transform this linear system into another easier equivalent system
\[
  \begin{aligned}
    x_1 - 3x_2 = 1 \\
    -x_1 + 5x_2 = 3
  \end{aligned}
\]
\textbf{Solution} - Add the first equation to the second equation
\[
  \begin{aligned}
    x_1 - 3x_2 = 1 \\
    2x_2 = 4
  \end{aligned}
\]
$x_2 = 2$ and $x_1 = 1 + 3(2) = 7$
\section{Matrices and Linear Systems}
\textbf{Definition} - An $m \times n$ matrix is a rectangular array of numbers with $m$ rows and $n$ columns. \\\\
\textbf{Example Matrices}
\[
  \begin{aligned}
    \begin{bmatrix}
      1 & 2 \\
      3 & 4
    \end{bmatrix} \quad
    \begin{bmatrix}
      1 & 2 & 3 \\
      2 & 1 & 4
    \end{bmatrix} \quad
    \begin{bmatrix}
      1 & 2 \\
      3 & 4 \\
      5 & 6
    \end{bmatrix} \quad
    \begin{bmatrix}
      1 + \sqrt{5}
    \end{bmatrix} \quad
    \begin{bmatrix}
      1 & -2 & 3 & 4 \\
      5 & -6 & 7 & 8 \\
      -9 & 10 & 11 & 12
    \end{bmatrix}
  \end{aligned}
\]
In terms of the entries of $A$:
\[ A = \begin{bmatrix}
    a_11 & a_12 & \dots & a_1n \\
    a_21 & a_22 & \dots & a_2n \\
   \vdots & \vdots & \ddots & \vdots \\
    a_m1 & a_m2 & \dots & a_mn
  \end{bmatrix} \]
where $a_{ij}$ is in the $i$th row and $j$th column \\\\
\textbf{Definition} - For a linear system, we define the \textbf{coefficient} and \textbf{augmented} matrix as follows: \\\\
\textbf{Linear System}
\[
  \begin{aligned}
    a_{11}x_{1} + a_{12}x_{2} + \dots + a_{1n}x_n &= b_1 \\
    a_{21}x_{1} + a_{22}x_{2} + \dots + a_{2n}x_n &= b_2 \\
                                                  &\hspace{0.5cm} \vdots \\
    a_{m1}x_{1} + a_{m2}x_{2} + \dots + a_{mn}x_n &= b_m
  \end{aligned}
\]
\textbf{Coefficient Matrix}
\[
  \begin{bmatrix}
    a_{11} & a_{12} & \dots & a_{1n} \\
    a_{21} & a_{22} & \dots & a_{2n} \\
    \dots & \dots & \ddots & \dots \\
    a_{m1} & a_{m2} & \dots & a_{mn}
  \end{bmatrix}
\]
\textbf{Augmented Matrix}
\[
\left[
\begin{array}{cccc|c}
  a_{11} & a_{12} & \dots & a_{1n} & b_1 \\
  a_{21} & a_{22} & \dots & a_{2n} & b_2 \\
  \vdots & \vdots & \ddots & \vdots & \vdots \\
  a_{m1} & a_{m2} & \dots & a_{mn} & b_m
\end{array}
\right]
\]
\subsection{Example Problem}
Determine the coefficient matrix and augmented matrix of the linear system
\[
  \begin{aligned}  
  x_1 - 3x_2 &= 1 \\
  -x_1 + 5x_2 &= 3
  \end{aligned}
\]
\textbf{Solution} - The coefficient matrix would be
\[
  \begin{bmatrix}
    1 & -3 \\
    -1 & 5
  \end{bmatrix}
\]
and the augmented matrix would be
\[
  \left[
  \begin{array}{cc|c}
    1 & -3 & 1 \\
    -1 & 5 & 3
  \end{array}
\right]
\]
\subsection{Elementary Row Operation}
An \textbf{elementary row operation} is one of the following
\begin{itemize}
  \item Replacement - add a multiple of one row to another row: $R_i \rightarrow R_i + cR_j$, where $i \neq j$.
  \item Interchange - Interchange two rows: $R_i \leftrightarrow R_j$
  \item Scaling - Multiply all entries in a row by a nonzero constant: $R_i \rightarrow cR_i$, where $c \neq 0$
\end{itemize}
\subsubsection{Example Problem}
Give several examples of elementary row operations \\\\
\textbf{Solution}
\begin{itemize}
  \item Replacement
  \[  
    \begin{align}
      &R_2 \rightarrow R_2 + 3R_1 \\
      &\begin{bmatrix}
        1 & 0 \\
        0 & 1
      \end{bmatrix} \rightarrow
      \begin{bmatrix}
        1 & 0 \\
        3 & 1
      \end{bmatrix}
    \end{align}
  \]
  \item Interchange
  \[
    \begin{align}
      R_1 &\leftrightarrow R_3 \\  
      \begin{bmatrix}
        1 & 0 \\
        0 & 1 \\
        2 & 3
      \end{bmatrix} &\rightarrow
      \begin{bmatrix}
        2 & 3 \\
        0 & 1 \\
        1 & 0
      \end{bmatrix}
    \end{align}
  \]
  \item Scaling
  \[
    \begin{align}
      R_2 &\rightarrow 3R_2 \\    
      \begin{bmatrix}
        1 & 0 \\
        0 & 1
      \end{bmatrix} &\rightarrow 
      \begin{bmatrix}
        1 & 0 \\
        0 & 3
      \end{bmatrix}
    \end{align}
  \]
\end{itemize}
Elementary row operations can undo or \textbf{reverse} each other. For example, the elementary row operation $R_3 \rightarrow R_3 - 3R_1$ reverses the row operation of $R_3 \rightarrow R_3 + 3R_1$
\[
  \begin{aligned}
    &R_3 \rightarrow R_3 + 3R_1 \\
    &\begin{bmatrix}
      1 & 0 & 0 \\
      0 & 1 & 0 \\
      0 & 0 & 1
    \end{bmatrix} \rightarrow
    \begin{bmatrix}
      1 & 0 & 0 \\
      0 & 1 & 0 \\
      3 & 0 & 1
    \end{bmatrix}
  \end{aligned} \\
\] \\
\[
  \begin{aligned}
    &R_3 \rightarrow R_3 - 3R_1 \\
    &\begin{bmatrix}
      1 & 0 & 0 \\
      0 & 1 & 0 \\
      3 & 0 & 1
    \end{bmatrix} \rightarrow
    \begin{bmatrix}
      1 & 0 & 0 \\
      0 & 1 & 0 \\
      0 & 0 & 1
    \end{bmatrix}
  \end{aligned}
\] \\
Every row operation is reversible. Above showed an example of reversing the replacement operator. 
Similarly, the scaling operator $R_2 \rightarrow cR_2$ is reversed by the scaling operator $R_2 \rightarrow \frac{1}{c}R_2$. 
Row interchange $R_1 \leftrightarrow R_2$ is reversible by performing it twice. \\\\
Two matrices are \textbf{row equivalent} if one matrix can be transformed into the other matrix by a sequence of elementary row operations.
\subsection{Theorem 1}
If the augmented matrices of two linear systems are row equivalent, then the two systems have the same solution set.
\section{Echelon forms of matrices}
\textbf{Definition} - A matrix is in echelon form or \textbf{row echelon form} when
\begin{enumerate}
  \item All \textbf{nonzero rows} (rows with at least one nonzero element) are above any rows of all zeroes
  \item The \textbf{leading entry} (the first nonzero number from the left) of a nonzero row is always strictly to the right of the leading entry of the row above it.
\end{enumerate}
\subsection{Example}
The following matrices achieve row echelon form
\[
  \begin{bmatrix}
    2 & -2 & 3 \\
    0 & 5 & 0 \\
    0 & 0 & \frac{5}{2}
  \end{bmatrix} \quad
  \begin{bmatrix}
    0 & 1 & 3 \\
    0 & 0 & \sqrt{2} \\
    0 & 0 & 0
  \end{bmatrix} \quad
  \begin{bmatrix}
    3 & 1 & 2 & 0 & 5 \\
    0 & 2 & 0 & 1 & 4 \\
    0 & 0 & 0 & 0 & 0 \\
    0 & 0 & 0 & 0 & 0
  \end{bmatrix}
\]
\subsection{RREF - Row Reduced Echelon Form}
A matrix is in \textbf{reduced row echelon form} (RREF) if it is in row echelon form \textbf{and}
\begin{itemize}
  \item The leading entry in each nonzero row is 1
  \item Each leading entry is the only nonzero entry in its column
\end{itemize}
\subsubsection{Examples}
The following matrices are in RREF
\[
  \begin{bmatrix}
    1 & 0 & 0 \\
    0 & 1 & 0 \\
    0 & 0 & 1
  \end{bmatrix} \quad
  \begin{bmatrix}
    0 & 1 & 3 & 0 & 0 & 2 & 5 & 0 & 0 & 6 \\
    0 & 0 & 0 & 1 & 0 & 1 & \frac{1}{2} & 0 & 0 & -2 \\
    0 & 0 & 0 & 0 & 1 & -3 & 4 & 0 & 0 & 5 \\
    0 & 0 & 0 & 0 & 0 & 0 & 0 & 1 & 0 & 0 \\
    0 & 0 & 0 & 0 & 0 & 0 & 0 & 0 & 1 & 1
  \end{bmatrix}
\]
\subsection{Theorem 1}
Each matrix is row-equivalent to one and only one matrix in reduced echelon form. \\\\ 
\textbf{Definition} - We say a matrix $B$ is the \textbf{reduced echelon form} (RREF) of a matrix if $A$ and $B$ are row-equivalent and $B$ is in reduced echelon form.
\subsubsection{Example Problem}
Is each matrix also row-equivalent to one and only one matrix in echelon form? \\\\
\textbf{Solution} - No! For example, $\begin{bmatrix}
  1 & 0 \\
  0 & 1
\end{bmatrix}$ and $\begin{bmatrix}
  1 & 2 \\
  0 & 1
\end{bmatrix}$ are row-equivalent and both in echelon form.
\subsection{Calculating RREF}
Find the rref of matrix $\begin{bmatrix}
  3 & -9 & 12 & -9 & 6 & 15 \\
  3 & -7 & 8 & -5 & 8 & 9
\end{bmatrix}$ \\\\
\textbf{Solution} - to achieve RREF, the leading entry of each nonzero row needs to be 1 and each leading entry is the only nonzero entry in the column 
\begin{enumerate}
  \item $R_2 \rightarrow R_2 - R_1$
    \[
      \begin{bmatrix}
        3 & -9 & 12 & -9 & 6 & 15 \\
        0 & 2 & -4 & 4 & 2 & -6
      \end{bmatrix}
    \]
  \item $R_1 \rightarrow \frac{1}{3}R_1$, $R_2 \rightarrow \frac{1}{2}R_2$
    \[
      \begin{bmatrix}
        1 & -3 & 4 & -3 & 2 & 5 \\
        0 & 1 & -2 & 2 & 1 & -3
      \end{bmatrix}
    \]
  \item $R_1 \rightarrow R_1 + 3R_2$
    \[
      \begin{bmatrix}
        1 & 0 & -2 & 3 & 5 & -4 \\
        0 & 1 & -2 & 2 & 1 & -3
      \end{bmatrix}
    \]
\end{enumerate} \\\\
\subsection{Pivot Position}
The position of a leading entry in an echelon form of a matrix. A \textbf{pivot column} is a column that contains a pivot position
\subsubsection{Example Problem}
Locate the pivot columns of the following matrix
\[
  A = \begin{bmatrix}
    0 & -3 & -6 & 4 & 9 \\
    -1 & -2 & -1 & 3 & 1 \\
    1 & 4 & 5 & -9 & -7
  \end{bmatrix}
\]
\textbf{Solution}
\begin{enumerate}
  \item $R_1 \leftrightarrow R_3$
    \[
      \begin{bmatrix}
        1 & 4 & 5 & -9 & -7 \\
        -1 & -2 & -1 & 3 & 1 \\
        0 & -3 & -6 & 4 & 9
      \end{bmatrix}
    \]
  \item $R_2 \rightarrow R_2 + R_1$
    \[
      \begin{bmatrix}
        1 & 4 & 5 & -9 & -7 \\
        0 & 2 & 4 & -6 & -6 \\
        0 & -3 & -6 & 4 & 9
      \end{bmatrix}
    \]
  \item $R_3 \rightarrow R_3 + 1.5R_2$
    \[
      \begin{bmatrix}          
        1 & 4 & 5 & -9 & -7 \\
        0 & 2 & 4 & -6 & -6 \\
        0 & 0 & 0 & -5 & 0
      \end{bmatrix}
    \]
\end{enumerate}
The columns of \textbf{1}, \textbf{2}, and \textbf{4} are the pivot columns of $A$.
\subsection{Pivot Variables}
\textbf{Basic Variable} (Pivot Variable) - A variable that corresponds to a pivot column in the coefficient matrix of a linear system. A \textbf{free variable} is a variable that is not a pivot variable. 
\subsubsection{Example Problem}
Consider the augmented matrix and system. Determine the basic and free variables.
\[
  \left[
  \begin{array}{ccccc|c}
    1 & 6 & 0 & 3 & 0 & 0 \\
    0 & 0 & 1 & -8 & 0 & 5 \\
    0 & 0 & 0 & 0 & 1 & 7
  \end{array}
  \right] \quad
  \begin{aligned}
    x_1 + 6x_2 + 3x_4 &= 0 \\
    x_3 - 8x_4 &= 5 \\
    x_5 &= 7
  \end{aligned}
\]
\textbf{Solution} -
The first, third and fifth columns are pivot columns. Therefore, $x_1$, $x_3$, and $x_5$ are basic variables and $x_2$, $x_4$ are free variables.
